\documentclass[11pt]{article}

%%%%%%%%%%%%%%%%%%%%%%%%%%%%%%%%%%%%%%%%
% Packages
%%%%%%%%%%%%%%%%%%%%%%%%%%%%%%%%%%%%%%%%

\usepackage{oke-header-math}

%%%%%%%%%%%%%%%%%%%%%%%%%%%%%%%%%%%%%%%%
% Mathematics
%%%%%%%%%%%%%%%%%%%%%%%%%%%%%%%%%%%%%%%%



%%%%%%%%%%%%%%%%%%%%%%%%%%%%%%%%%%%%%%%%
% Title
%%%%%%%%%%%%%%%%%%%%%%%%%%%%%%%%%%%%%%%%

\title{tfConstrainedGauss Python package}
\author{Oliver K. Ernst}

%%%%%%%%%%%%%%%%%%%%%%%%%%%%%%%%%%%%%%%%
% Begin document
%%%%%%%%%%%%%%%%%%%%%%%%%%%%%%%%%%%%%%%%

\begin{document}

\maketitle

This package implements two methods for finding a sparse precision matrix with a given structure from a given covariance matrix.

%%%%%%%%%%%%%%%%%%%%%%%%%%%%%%%%%%%%%%%%%%%%%%%%%%%%%%%%%%%%%%%%%%%%%%%%%%%%%%%%
%%%%%%%%%%%%%%%%%%%%%%%%%%%%%%%%%%%%%%%%%%%%%%%%%%%%%%%%%%%%%%%%%%%%%%%%%%%%%%%%

\section{Identity-based method}

%%%%%%%%%%%%%%%%%%%%%%%%%%%%%%%%%%%%%%%%%%%%%%%%%%%%%%%%%%%%%%%%%%%%%%%%%%%%%%%%
%%%%%%%%%%%%%%%%%%%%%%%%%%%%%%%%%%%%%%%%%%%%%%%%%%%%%%%%%%%%%%%%%%%%%%%%%%%%%%%%

Given an $n\times n$ covariance matrix, here of size $n=3$:
%---------------
\begin{equation}
\Sigma = \begin{pmatrix}
	c_{11} & c_{12} & c_{13} \\
	c_{12} & c_{22} & c_{23} \\
	c_{13} & c_{23} & c_{33}
\end{pmatrix}
\end{equation}
%---------------
and given the structure of the precision matrix (i.e. given the Gaussian graphical model), for example:
%---------------
\begin{equation}
P = \begin{pmatrix}
	p_{11} & p_{12} & 0 \\
	p_{12} & p_{22} & p_{23} \\
	0 & p_{23} & p_{33}
\end{pmatrix}
\end{equation}
%---------------
(note that the diagonal elements are always non-zero), the goal is to find the elements of the precision matrix by:
%---------------
\begin{equation}
P^* = \underset{P}{\text{argmin}} |P \Sigma - I|
\end{equation}
%---------------
where $I$ is the identity.

The advantage of this approach is that it does not require calculating the inverse of any matrix, particularly important for large $n$.

The disadvantage of this approach is that the solution found for $P$ may not yield a covariance matrix $P^{-1}$ whose individual elements are close to those of $\Sigma$. That is, while $P \Sigma$ may be close to the identity, there are likely errors in every single element of $P^{-1}$.

%%%%%%%%%%%%%%%%%%%%%%%%%%%%%%%%%%%%%%%%%%%%%%%%%%%%%%%%%%%%%%%%%%%%%%%%%%%%%%%%
%%%%%%%%%%%%%%%%%%%%%%%%%%%%%%%%%%%%%%%%%%%%%%%%%%%%%%%%%%%%%%%%%%%%%%%%%%%%%%%%

\section{MaxEnt-based method}

%%%%%%%%%%%%%%%%%%%%%%%%%%%%%%%%%%%%%%%%%%%%%%%%%%%%%%%%%%%%%%%%%%%%%%%%%%%%%%%%
%%%%%%%%%%%%%%%%%%%%%%%%%%%%%%%%%%%%%%%%%%%%%%%%%%%%%%%%%%%%%%%%%%%%%%%%%%%%%%%%

Given the structure of the $n\times n$ precision matrix (i.e. given the Gaussian graphical model), for example:
%---------------
\begin{equation}
P = \begin{pmatrix}
	p_{11} & p_{12} & 0 \\
	p_{12} & p_{22} & p_{23} \\
	0 & p_{23} & p_{33}
\end{pmatrix}
\end{equation}
%---------------
(note that the diagonal elements are always non-zero), and given the covariances for corresponding to every \textit{non-zero} entry in $P$, i.e. given:
%---------------
\begin{equation}
c_{11}, c_{12}, c_{22}, c_{23}, c_{33}
\end{equation}
%---------------
the goal is to find the elements of $P$. In other words, every unique element $(i,j)$ of the $n\times n$ symmetric matrix has a given constraint, either to a value in the covariance matrix, or a zero entry in the precision matrix.

This is a maximum entropy (MaxEnt) setup. The elements of the precision matrix $p_{ij}$ are directly the interactions in the Gaussian graphical model; the moments they control in a MaxEnt sense are the covariances $c_{ij}$.

The problem can be solved in a number of ways, for example using Boltzmann machine learning, where we minimize:
%---------------
\begin{equation}
P^* = \underset{P}{\text{min}} \, \dkl = \underset{P}{\text{min}} \sum_n p(n) \ln \frac{p(n)}{\pt(n)}
\end{equation}
%---------------
where $p(n)$ is the (unknown) data distribution that gave rise to the given covariances $c_{ij}$ and $\pt(n)$ is the Gaussian with precision matrix $P$. The gradients that result are the wake sleep phase:
%---------------
\begin{equation}
\Delta p_{ij} \propto c_{ij} - (P^{-1})_{ij}
\end{equation}
%---------------

In TensorFlow, we minimize the MSE loss for the individual terms, which results in the same first order gradients:
%---------------
\begin{equation}
\underset{P}{\text{min}} \sum_{ij} \Big | \Big | c_{ij} - (P^{-1})_{ij} \Big | \Big |_2
\end{equation}
%--------------- 

To learn each element of the covariance matrix with equal importance, we can use a weighted MSE loss:
%---------------
\begin{equation}
\underset{P}{\text{min}} \sum_{ij} w_{ij} \Big | \Big | c_{ij} - (P^{-1})_{ij} \Big | \Big |_2
\end{equation}
%---------------
where
%---------------
\begin{equation}
w_{ij} = c_{ij}^{-2}
\end{equation}
%---------------

\end{document}


